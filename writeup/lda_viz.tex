%% CMSC 691 Final Project.
%% AUTHORS - Ashwinkumar Ganesan.
%%           Kiante Branley

\documentclass[10pt,journal,compsoc]{IEEEtran}
% \documentclass[10pt,journal,compsoc]{../sty/IEEEtran}

% To go back to Times Roman, you can use this code:
%\renewcommand{\rmdefault}{ptm}\selectfont

\usepackage{ragged2e}

% *** CITATION PACKAGES ***
%
\ifCLASSOPTIONcompsoc
  \usepackage[nocompress]{cite}
\else
  \usepackage{cite}
\fi

% *** GRAPHICS RELATED PACKAGES ***
%
\ifCLASSINFOpdf
  % \usepackage[pdftex]{graphicx}
\else
  % \usepackage[dvips]{graphicx}
  % \graphicspath{{../eps/}}
  % \DeclareGraphicsExtensions{.eps}
\fi

% *** MATH PACKAGES ***
%
%\usepackage[cmex10]{amsmath}
%\interdisplaylinepenalty to 10000
%\interdisplaylinepenalty=2500

% *** SPECIALIZED LIST PACKAGES ***
%\usepackage{acronym}
% \renewcommand{\IEEEiedlistdecl}{\IEEEsetlabelwidth{SONET}}
% \begin{acronym}
%
% \end{acronym}
% \renewcommand{\IEEEiedlistdecl}{\relax}
%\usepackage{algorithmic}

% *** ALIGNMENT PACKAGES ***
%
%\usepackage{array}
%\usepackage{mdwmath}
%\usepackage{mdwtab}
%\usepackage{eqparbox}

% *** SUBFIGURE PACKAGES ***
%\ifCLASSOPTIONcompsoc
%  \usepackage[caption=false,font=footnotesize,labelfont=sf,textfont=sf]{subfig}
%\else
%  \usepackage[caption=false,font=footnotesize]{subfig}
%\fi
% The latest version and documentation of subfig.sty can be obtained at:
% http://www.ctan.org/tex-archive/macros/latex/contrib/subfig/


% *** FLOAT PACKAGES ***
%
%\usepackage{fixltx2e}
%\usepackage{stfloats}
%
% \usepackage{dblfloatfix}

%\ifCLASSOPTIONcaptionsoff
%  \usepackage[nomarkers]{endfloat}
% \let\MYoriglatexcaption\caption
% \renewcommand{\caption}[2][\relax]{\MYoriglatexcaption[#2]{#2}}
%\fi

% \let\MYorigsubfloat\subfloat
% \renewcommand{\subfloat}[2][\relax]{\MYorigsubfloat[]{#2}}

% *** PDF, URL AND HYPERLINK PACKAGES ***
%
%\usepackage{url}

% NOTE: PDF thumbnail features are not required in IEEE papers
%       and their use requires extra complexity and work.
%\ifCLASSINFOpdf
%  \usepackage[pdftex]{thumbpdf}
%\else
%  \usepackage[dvips]{thumbpdf}
%\fi
% 
% thumbpdf filename.pdf 
%

% NOTE: PDF hyperlink and bookmark features are not required in IEEE
%       papers and their use requires extra complexity and work.
% *** IF USING HYPERREF BE SURE AND CHANGE THE EXAMPLE PDF ***
% *** TITLE/SUBJECT/AUTHOR/KEYWORDS INFO BELOW!!           ***
\newcommand\MYhyperrefoptions{bookmarks=true,bookmarksnumbered=true,
pdfpagemode={UseOutlines},plainpages=false,pdfpagelabels=true,
colorlinks=true,linkcolor={black},citecolor={black},urlcolor={black},
pdftitle={CMSC 691 Project Report},%<!CHANGE!
pdfsubject={Typesetting},%<!CHANGE!
pdfauthor={Ashwinkumar Ganesan, Kiante Branley},%<!CHANGE!
pdfkeywords={DataViz, InfoViz, LDA, Parallel Coordinates, Treemap, Topic Modeling}}

%\ifCLASSINFOpdf
%\usepackage[\MYhyperrefoptions,pdftex]{hyperref}
%\else
%\usepackage[\MYhyperrefoptions,breaklinks=true,dvips]{hyperref}
%\usepackage{breakurl}
%\fi
% correct bad hyphenation here
\hyphenation{}

% Title.
\begin{document}
\title{Visualizing Topic Models Generated Using Latent Dirichlet Allocation}

% Author.
\author{Ashwinkumar~Ganesan,
        Kiante~Branley
\IEEEcompsocitemizethanks{\IEEEcompsocthanksitem Ashwinkumar Ganesan is with the Department
of Computer Science and Electrical, UMBC.

Kiante Branley is with the Department
of Computer Science and Electrical, UMBC.
\protect\\
E-mail: \{gashwin1, bran4\}@umbc.edu}}

% The paper headers
\markboth{CMSC 691 Project Report}%
{Shell \MakeLowercase{\textit{et al.}}: }

% Abstract.
\IEEEtitleabstractindextext{%
\justify
\begin{abstract}
Topic Modeling is a set of statistical methods used to find "topics" in a given document corpus, where a \textit{topic} is defined as a word distribution. Topic Modeling opens the doors to a set of interesting questions such as the various topics in documents \& which documents are associated which kinds of topics. One of the problems with statistical methods is that the word distributions generated may not interpretable by the user. Our project, \textit{LDAExplore}, discusses the visualization designed, to look at topic models and how the user can interact with these statistical methods and provide feedback to improve the underlying model. Latent Dirichlet Allocation (LDA) is one of the basic methods to find such hidden topics that has been used in the project. To validate our design, we have used the \textit{abstract}'s of 322 \textit{Information Visualization} papers, where every abstract is considered a \textit{document} and generated topics which are then explored by users.
\end{abstract}

% Note that keywords are not normally used for peerreview papers.
\begin{IEEEkeywords}
DataViz, InfoViz, LDA, Latent, Dirichlet Allocation, Parallel Coordinates, Treemap, Topic Modeling.
\end{IEEEkeywords}}

% make the title area
\maketitle
\IEEEdisplaynontitleabstractindextext
%
% \ifCLASSOPTIONpeerreview
% \begin{center} \bfseries EDICS Category: 3-BBND \end{center}
% \fi
%
\IEEEpeerreviewmaketitle


\ifCLASSOPTIONcompsoc
\IEEEraisesectionheading{\section{Introduction}\label{sec:introduction}}
\else
\section{Introduction}
\label{sec:introduction}
\fi

% \LaTeX\
\IEEEPARstart{A} large body of human knowledge has traditionally been stored in the form of documents. These documents maybe in the form of books, magazines, journals. With the advent of the digital age, more and more content is stored digitally and with the cloud computing becoming increasingly common, we stored a lot of information online. We have websites that present this knowledge online in the form of HTML pages and are looking at this information being presented on mobile platforms such as IOS and Android in the form of apps. This growing knowledge base brings with it a unique set of challenges such as searching through a large set of documents for a specific piece of information, grouping similar documents together, looking at how these documents and their underlying content change over time. These changes could be the common theme or the language that is utilized in these documents.\\

\textit{Topic Modelling} tries to automate the process of extracting topics from documents and annotate them with semantic information [2]. It is a set of statistical algorithms that extract correlated words from documents. These extracted word sets are called \textit{Topics}. They are later annotated with semantic information, so that they are easily understood by people. For e.g. consider a a word set extracted such as \textit{visualization, sets, clusters, infoviz, interfaces} from a document then we have a general notion that this word set represents the topic \textit{Information Visualization}. In this example, the "topic" generated using a topic modelling algorithm is the word set and \textit{Information Visualization} is the semantic "topic name" annotated by the user. Latent Dirichlet Allocation (LDA) [1] is one of the common methods to perform topic modelling on a given corpus of documents. \\

LDA generates viz. two types of distributions i.e. the topic distribution for each document in the set and the word distribution for each topic. These distributions can be changed by tweaking some of the underlying parameters. Our project \textit{LDAExplore}, tries to give visual cues about how these distributions look, and how the topics and documents are interrelated at the corpus level and for groups or individual documents. Also, it gives the users the ability query documents for specific words based on topics and see how correlated a topic is to a document. Some of the main concerns while creating the design are that visual should be able to work with a large set of documents while providing the ability to see individual and group relations. The number of topics, though, is considered to be a smaller set as compared to the number of documents.

\section{Related Work}
\label{sec:RelatedWork}
There are quite a few ways of looking that the problem of how to represent distributions to show their interrelations. Some of the methods include looking at distributions graphs or sets [5]. In these visualizations, the distributions are converted to sets transforming them to a hierarchy of nodes. A standard technique to represent sets is to create an \textit{Euler} diagram [5] or a variant of it. Region-based overlays can be used such as \textit{Bubble Sets} and \textit{Texture Splatting} or Line-based overlays like \textit{LineSets} and \textit{Kelp Diagrams} [5]. Some of the other methods include \textit{Node-Link} diagrams and \textit{Force-Directed} graphs. Standard Node-link diagrams include \textit{Jigsaw}, Anchored maps and \textit{PivotPaths}. These design patterns invariably limit the number of nodes that can be represented in a visual without applying some aggregation method which clusters nodes or edges together.

Topic-based, interactive visual analysis tool (TIARA) [3] shows topic distributions across documents across time. This helps 

\subsection{Subsection Heading Here}
Subsection text here.

% needed in second column of first page if using \IEEEpubid
%\IEEEpubidadjcol

\subsubsection{Subsubsection Heading Here}
Subsubsection text here.


% An example of a floating figure using the graphicx package.
% Note that \label must occur AFTER (or within) \caption.
% For figures, \caption should occur after the \includegraphics.
% Note that IEEEtran v1.7 and later has special internal code that
% is designed to preserve the operation of \label within \caption
% even when the captionsoff option is in effect. However, because
% of issues like this, it may be the safest practice to put all your
% \label just after \caption rather than within \caption{}.
%
% Reminder: the "draftcls" or "draftclsnofoot", not "draft", class
% option should be used if it is desired that the figures are to be
% displayed while in draft mode.
%
%\begin{figure}[!t]
%\centering
%\includegraphics[width=2.5in]{myfigure}
% where an .eps filename suffix will be assumed under latex, 
% and a .pdf suffix will be assumed for pdflatex; or what has been declared
% via \DeclareGraphicsExtensions.
%\caption{Simulation results for the network.}
%\label{fig_sim}
%\end{figure}

% Note that IEEE typically puts floats only at the top, even when this
% results in a large percentage of a column being occupied by floats.
% However, the Computer Society has been known to put floats at the bottom.


% An example of a double column floating figure using two subfigures.
% (The subfig.sty package must be loaded for this to work.)
% The subfigure \label commands are set within each subfloat command,
% and the \label for the overall figure must come after \caption.
% \hfil is used as a separator to get equal spacing.
% Watch out that the combined width of all the subfigures on a 
% line do not exceed the text width or a line break will occur.
%
%\begin{figure*}[!t]
%\centering
%\subfloat[Case I]{\includegraphics[width=2.5in]{box}%
%\label{fig_first_case}}
%\hfil
%\subfloat[Case II]{\includegraphics[width=2.5in]{box}%
%\label{fig_second_case}}
%\caption{Simulation results for the network.}
%\label{fig_sim}
%\end{figure*}
%
% Note that often IEEE papers with subfigures do not employ subfigure
% captions (using the optional argument to \subfloat[]), but instead will
% reference/describe all of them (a), (b), etc., within the main caption.
% Be aware that for subfig.sty to generate the (a), (b), etc., subfigure
% labels, the optional argument to \subfloat must be present. If a
% subcaption is not desired, just leave its contents blank,
% e.g., \subfloat[].


% An example of a floating table. Note that, for IEEE style tables, the
% \caption command should come BEFORE the table and, given that table
% captions serve much like titles, are usually capitalized except for words
% such as a, an, and, as, at, but, by, for, in, nor, of, on, or, the, to
% and up, which are usually not capitalized unless they are the first or
% last word of the caption. Table text will default to \footnotesize as
% IEEE normally uses this smaller font for tables.
% The \label must come after \caption as always.
%
%\begin{table}[!t]
%% increase table row spacing, adjust to taste
%\renewcommand{\arraystretch}{1.3}
% if using array.sty, it might be a good idea to tweak the value of
% \extrarowheight as needed to properly center the text within the cells
%\caption{An Example of a Table}
%\label{table_example}
%\centering
%% Some packages, such as MDW tools, offer better commands for making tables
%% than the plain LaTeX2e tabular which is used here.
%\begin{tabular}{|c||c|}
%\hline
%One & Two\\
%\hline
%Three & Four\\
%\hline
%\end{tabular}
%\end{table}


% Note that the IEEE does not put floats in the very first column
% - or typically anywhere on the first page for that matter. Also,
% in-text middle ("here") positioning is typically not used, but it
% is allowed and encouraged for Computer Society conferences (but
% not Computer Society journals). Most IEEE journals/conferences use
% top floats exclusively. 
% Note that, LaTeX2e, unlike IEEE journals/conferences, places
% footnotes above bottom floats. This can be corrected via the
% \fnbelowfloat command of the stfloats package.




\section{Conclusion}
The conclusion goes here.





% if have a single appendix:
%\appendix[Proof of the Zonklar Equations]
% or
%\appendix  % for no appendix heading
% do not use \section anymore after \appendix, only \section*
% is possibly needed

% use appendices with more than one appendix
% then use \section to start each appendix
% you must declare a \section before using any
% \subsection or using \label (\appendices by itself
% starts a section numbered zero.)
%


\appendices
\section{Proof of the First Zonklar Equation}
Appendix one text goes here.

% you can choose not to have a title for an appendix
% if you want by leaving the argument blank
\section{}
Appendix two text goes here.


% use section* for acknowledgment
\ifCLASSOPTIONcompsoc
  % The Computer Society usually uses the plural form
  \section*{Acknowledgments}
\else
  % regular IEEE prefers the singular form
  \section*{Acknowledgment}
\fi


The authors would like to thank...


% Can use something like this to put references on a page
% by themselves when using endfloat and the captionsoff option.
\ifCLASSOPTIONcaptionsoff
  \newpage
\fi



% trigger a \newpage just before the given reference
% number - used to balance the columns on the last page
% adjust value as needed - may need to be readjusted if
% the document is modified later
%\IEEEtriggeratref{8}
% The "triggered" command can be changed if desired:
%\IEEEtriggercmd{\enlargethispage{-5in}}

% references section

% can use a bibliography generated by BibTeX as a .bbl file
% BibTeX documentation can be easily obtained at:
% http://www.ctan.org/tex-archive/biblio/bibtex/contrib/doc/
% The IEEEtran BibTeX style support page is at:
% http://www.michaelshell.org/tex/ieeetran/bibtex/
%\bibliographystyle{IEEEtran}
% argument is your BibTeX string definitions and bibliography database(s)
%\bibliography{IEEEabrv,../bib/paper}
%
% <OR> manually copy in the resultant .bbl file
% set second argument of \begin to the number of references
% (used to reserve space for the reference number labels box)

\begin{thebibliography}{1}
\bibitem aBlei, David M., Andrew Y. Ng, and Michael I. Jordan. "Latent dirichlet allocation." the Journal of machine Learning research 3 (2003): 993-1022.
\bibitem aBlei, David M. "Probabilistic topic models." Communications of the ACM 55.4 (2012): 77-84.
\bibitem aPan, Shimei, et al. "Optimizing temporal topic segmentation for intelligent text visualization." Proceedings of the 2013 international conference on Intelligent user interfaces. ACM, 2013.
\bibitem aYang, Yi, et al. "Active Learning with Constrained Topic Model."
\bibitem aAlsallakh, Bilal, et al. "Visualizing Sets and Set-typed Data: State-of-the-Art and Future Challenges (Supplementary Material)."
\bibitem aCui, W., Liu, S., Wu, Z., \& Wei, H. (2014). How hierarchical topics evolve in large text corpora.
\end{thebibliography}

% biography section
% 
% If you have an EPS/PDF photo (graphicx package needed) extra braces are
% needed around the contents of the optional argument to biography to prevent
% the LaTeX parser from getting confused when it sees the complicated
% \includegraphics command within an optional argument. (You could create
% your own custom macro containing the \includegraphics command to make things
% simpler here.)
%\begin{IEEEbiography}[{\includegraphics[width=1in,height=1.25in,clip,keepaspectratio]{mshell}}]{Michael Shell}
% or if you just want to reserve a space for a photo:

\begin{IEEEbiography}{Michael Shell}
Biography text here.
\end{IEEEbiography}

% if you will not have a photo at all:
\begin{IEEEbiographynophoto}{John Doe}
Biography text here.
\end{IEEEbiographynophoto}

% insert where needed to balance the two columns on the last page with
% biographies
%\newpage

\begin{IEEEbiographynophoto}{Jane Doe}
Biography text here.
\end{IEEEbiographynophoto}

% You can push biographies down or up by placing
% a \vfill before or after them. The appropriate
% use of \vfill depends on what kind of text is
% on the last page and whether or not the columns
% are being equalized.

%\vfill

% Can be used to pull up biographies so that the bottom of the last one
% is flush with the other column.
%\enlargethispage{-5in}



% that's all folks
\end{document}


